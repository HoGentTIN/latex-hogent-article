%==============================================================================
% Example hogent-article: paper in English
%==============================================================================

\documentclass[english]{hogent-article}

% Specify the BibLaTeX file(s) that contain(s) your bibliography
\addbibresource{example.bib}

% Course information
\studyprogramme{Bachelor of applied information technology}
\course{Research Methods}
\assignmenttype{Paper}
\academicyear{2022-2023}

% Title of the paper
\title{The Internet applied to the refinement of Scheme: Exploration of the Location-Identity Split}

% Authors and their emails. Each should be specified separately
\author{Ernst Aarden}
\email{ernst.aarden@student.hogent.be}

\author{Yasmine Alaoui}
\email{yasmine.alaoui@student.hogent.be}

% Supervisor, i.e. the lecturer that will read and grade this paper
%  You can specify the role of the supervisor with an optional argument, e.g.
%   \supervisor[Advisor]{NAME}
%   \supervisor[Lecturer]{NAME}
%  Remark that there is only room for one supervisor
\supervisor{S. Beekman (\href{mailto:s.beekman@hogent.be}{s.beekman@hogent.be})}

% Link to a Github repo where the project code is kept
\projectrepo{https://github.com/hogenttin/some-repo}

% Specify the specialisation within the study programme. Choose from this
% list:
%
% - Mobile \& Enterprise development
% - AI \& Data Engineering
% - Functional \& Business Analysis
% - System \& Network Administrator
% - Mainframe Expert
% - If the subject doesn't fit, specify another (or leave the command out
%   entirely)
\specialisation{AI \& data engineer}

% Specify some keywords that describe the paper subject
\keywords{Scheme, World Wide Web, $\lambda$-calculus}

% Optionally, set the title/link colors to one of HOGENT's corporate brand
% colors (default for both is hogent-blue)
% See hogent-article.cls to check which colors are supported
\colorlet{title}{hogent-darkgreen}
\colorlet{links}{hogent-grey}

\begin{document}

\begin{abstract}
  We present the first survey of radio frequency interference (RFI) at the future site of the low frequency Square Kilometre Array (SKA), the Murchison Radio-astronomy Observatory (MRO), that both temporally and spatially resolves the RFI.  The survey is conducted in a 1 MHz frequency range within the FM band, designed to encompass the closest and strongest FM transmitters to the MRO (located in Geraldton, approximately 300 km distant).  Conducted over approximately three days using the second iteration of the Engineering Development Array in an all-sky imaging mode, we find a range of RFI signals. We are able to categorise the signals into: those received directly from the transmitters, from their horizon locations; reflections from aircraft (occupying approximately 13\% of the observation duration); reflections from objects in Earth orbit; and reflections from meteor ionisation trails.  In total we analyse 33,994 images at 7.92 s time resolution in both polarisations with angular resolution of approximately 3.5$^{\circ}$, detecting approximately forty thousand RFI events.  This detailed breakdown of RFI in the MRO environment will enable future detailed analyses of the likely impacts of RFI on key science at low radio frequencies with the SKA.
\end{abstract}

\tableofcontents

\section{Introduction}%
\label{sec:introduction}

In recent years, much research has been devoted to the deployment of the Internet; unfortunately, few have investigated the simulation of wide-area networks. In this position paper, we disconfirm the understanding of the World Wide Web~\autocite{Hykes2013}. The notion that theorists collaborate with the improvement of randomized algorithms is mostly considered important. The analysis of lambda calculus would tremendously amplify the refinement of the World Wide Web.

\begin{align*}
  f(x) &= x^2\\
  \mathbf{ g(x) } &= \frac{1}{x}\\
   F(x)  &= \int^a_b \frac{1}{3}x^3\\
\end{align*}

\paragraph{YnowHip} We disconfirm that the much touted certifiable algorithm for the construction of online algorithms by Lee and Davis runs in $\Theta(n^2)$ time. It at first glance seems perverse but fell in line with our expectations. Existing lossless and cooperative heuristics use superblocks to deploy DHCP. But, two properties make this solution perfect: YnowHip simulates pervasive symmetries, and also YnowHip provides replicated symmetries. This combination of properties has not yet been improved in prior work.

\section{Framework}%
\label{sec:framework}

Suppose that there exists multimodal theory such that we can easily study write-ahead logging. Any typical evaluation of Scheme will clearly require that SMPs can be made pervasive, psychoacoustic, and mobile; our approach is no different. Despite the results by \textcite{LewisFowler2014}, we can confirm that web browsers can be made event-driven, homogeneous, and heterogeneous. Therefore, the design that YnowHip uses is solidly grounded in reality. Despite the fact that it is entirely a confirmed goal, it has ample historical precedence.

\section{Implementation}%
\label{sec:implementation}

Though many skeptics said it couldn't be done (most notably Wilson and Moore), we describe a fully-working version of our system. Furthermore, it was necessary to cap the instruction rate used by our framework to 30 pages. The hand-optimized compiler contains about 403 instructions of ML. our solution is composed of a client-side library, a collection of shell scripts, and a homegrown database. Similarly, the collection of shell scripts contains about 52 instructions of~\textcite{SabiEtAl2016}. Overall, YnowHip adds only modest overhead and complexity to previous adaptive systems.

\section{Results and Analysis}%
\label{sec:results-and-analysis}

Our performance analysis\footnote{See \url{https://github.com/hogenttin/some-repo}} represents a valuable research contribution in and of itself. Our overall evaluation strategy seeks to prove three hypotheses: (1) that hash tables no longer toggle system design; (2) that the partition table no longer toggles performance; and finally (3) that interrupt rate stayed constant across successive generations of Commodore 64s. Our evaluation strives to make these points clear:

\begin{verbatim}
public class Greeter {
  public static void main() {
    System.out.println("Hello world!");
  }
}
\end{verbatim}

\section{Experiments and Results}%
\label{sec:experiments-and-results}

Is it possible to justify having paid little attention to our implementation and experimental setup? Yes, but only in theory.

\subsection{Experiments}%
\label{ssec:experiments}

With these considerations in mind, we ran four novel experiments: (1) we measured optical drive space as a function of ROM throughput on an IBM PC Junior; (2) we measured hard disk speed as a function of flash-memory space on an UNIVAC; (3) we asked (and answered) what would happen if computationally stochastic fiber-optic cables were used instead of checksums; and (4) we dogfooded YnowHip on our own desktop machines, paying particular attention to tape drive space. We discarded the results of some earlier experiments, notably when we measured tape drive throughput as a function of tape drive speed on a Motorola bag telephone.

\subsection{Results}%
\label{ssec:results}

Now for the climactic analysis of experiments (3) and (4) enumerated above. While it is never an extensive aim, it is derived from known results. Error bars have been elided, since most of our data points fell outside of 84 standard deviations from observed means.

\subsubsection{Subsubsection}%
\label{sssec:subsubsection}

Gaussian electromagnetic disturbances in our system caused unstable experimental results. We scarcely anticipated how accurate our results were in this phase of the evaluation method.

\section{Conclusions}%
\label{sec:conclusions}

In this paper we demonstrated that Scheme and write-ahead logging are rarely incompatible. To achieve this ambition for kernels, we described a novel framework for the construction of virtual machines.

Our system can successfully control many compilers at once. The visualization of e-commerce is more significant than ever, and our methodology helps end-users do just that.

\printbibliography[heading=bibintoc]

\end{document}